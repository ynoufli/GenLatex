\documentclass[a4paper,12pt]{article}
\usepackage[utf8]{inputenc}
\usepackage[T1]{fontenc}
\usepackage{geometry}
\usepackage{hyperref}
\usepackage{graphicx}
\geometry{margin=2cm}
\usepackage[colorlinks=true, linkcolor=blue, citecolor=blue, urlcolor=blue]{hyperref}


\title{\textbf{__Titre__}}

\begin{document}
\date{}
\maketitle

\section*{Introduction au sujet}

__introduction__

\section*{Mots-clés}

\begin{tabular}{|p{6cm}|p{6cm}|}
\hline
\textbf{Mots-clés} & \textbf{Keywords} \\
\hline
__motFR__ & __motANG__ \\
__motFR__ & __motANG__ \\
__MotOptionnel__ & __motOptionnel_ \\
__MotOptionnel__ & __motOptionnel_ \\
__MotOptionnel__ & __motOptionnel_ \\
\hline
\end{tabular}

\section*{Fondements mathématiques}

\subsection*{__Partie1__}

__explications__

\subsection*{__Partie2Optionnelle__}

__explicationsOptionnels__

\section*{Plan et exploration interactive}

__explorationDuPlan__


\section*{Références}

\begin{enumerate}
  \item \label{bib:first}
  __Nom__, \emph{__NomDeLarticle__}, 2004.  
  \url{__URLDeLarticle__Optionnel__}

  \item \label{bib:acerola}
  Acerola, \emph{Realistic Ocean Simulation with FFT}.  
  \url{https://www.youtube.com/@Acerola_t}

  \item \label{bib:gamper}
  Thomas Gamper, \emph{Ocean Surface Generation and Rendering}, Master’s thesis, TU Wien.  
  \url{https://www.cg.tuwien.ac.at/research/publications/2018/GAMPER-2018-OSG/GAMPER-2018-OSG-thesis.pdf}

  \item \label{bib:jump}
  Jump Trajectory, \emph{Ocean waves simulation with Fast Fourier Transform}.  
  \url{https://www.youtube.com/watch?v=kGEqaX4Y4bQ&t=597s}
\end{enumerate}


\end{document}
